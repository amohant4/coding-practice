\chapter{Data}

% \begin{chapquote}{Author's name, \textit{Source of this quote}}
% ``This is a quote and I don't know who said this.''
% \end{chapquote}

We stated above that Python supports the object-oriented programming paradigm. This means that Python considers data to be the focal point of the problem-solving process. In Python, as well as in any other object-oriented programming language, we define a class to be a description of what the data look like (the state) and what the data can do (the behavior). Classes are analogous to abstract data types because a user of a class only sees the state and behavior of a data item. Data items are called objects in the object-oriented paradigm. An object is an instance of a class.

%\begin{lstlisting}
%>>> print 'hello world'
%hello world
%\end{lstlisting}

\section{Built-in Data Types}
Python has two main built-in data types for integers and floating point numbers. These python classes are called 'int' and 'float'. The standard arithmetic operations, +, -, *, /, and ** (exponentiation), can be used with parentheses forcing the order of operations away from normal operator precedence. Other very useful operations are the remainder (modulo) operator, \%, and integer division, //. Note that when two integers are divided, the result is a floating point. The integer division operator returns the integer portion of the quotient by truncating any fractional part.
